
\documentclass{bmcart}

\usepackage[utf8]{inputenc} 

\def\includegraphic{}
\def\includegraphics{}
\startlocaldefs
\endlocaldefs

\begin{document}
	
	\begin{frontmatter}
		
		\dochead{Documento de Investigacion}
		
		\title{Educacion Ambiental}
		
		\author[
		]{\inits{HG}\fnm{Gutierrez Cruz} \snm{Jorman Eliud}}
		
	\end{frontmatter}
	
	\section*{Motivos de esta investigacion}
	Esta investigación es para crear conciencia e informar a las personas sobre este tema, como es que debe usarse y por que es necesario tener en cuenta en nuestra vida diaria.
	Es importante entonces pensar y saber que el mundo no nos pertenece, nos ha sido prestado para que vivamos en él y lo utilicemos con sabiduría. Y eso es lo que debemos hacer... vivir, no destruir.
	
	Pero también debemos proteger nuestro ambiente porque lo necesitamos. ¡Y mucho! Dependemos de él para existir. Nuestro planeta nos brinda todos los recursos naturales que necesitamos para alimentarnos, construir nuestras viviendas, tener luz, transportarnos, vestirnos, etc. Mira un segundo a tu alrededor... todo lo que ves - papel, lápiz, computadora, goma, etc.- se obtiene, directa o indirectamente, del ambiente, por lo cual es importante que aseguremos su capacidad de continuar proveyéndolos.
	
	Si destruimos el ambiente estaremos perjudicando a nosotros mismos, a nuestros hijos y a nuestros nietos. Cuidar el mundo es cuidarnos y esa es otra muy buena razón.
	
	\section*{Introduccion}
	Empecemos por describir lo que es la educacion ambiental:
	La educación ambiental es un proceso de formación  que permite la toma de conciencia de la importancia del medio ambiente, promueve en la ciudadanía el desarrollo de valores y nuevas actitudes que contribuyan al uso racional de los recursos naturales y a la solución de los problemas ambientales que enfrentamos en nuestra ciudad.
	
	\section*{Objetivos de la educacion ambiental}
	Los objetivos de la educación ambiental se encuentran íntimamente relacionados y cada uno de ellos depende del anterior. Son pasos que deben ir alcanzándose gradualmente para lograr la formación del individuo hacia el desarrollo sustentable. Dichos objetivos son:
	Conciencia, que se logra mediante la enseñanza al aire libre, la realización de campamentos, la organización de debates, distintos ejercicios de sensibilización, etc.
	Conocimientos sobre la realidad ambiental alcanzados recurriendo a estudios de campo, aplicación y desarrollo de modelos, simulaciones, investigaciones, redes conceptuales, entre otros.
	Actitudes vinculadas a las formas de percepción de la realidad ambiental y el desarrollo de la autoconciencia.
	Aptitudes y habilidades, logradas mediante el trabajo de campo, la realización de experiencias de laboratorio, la recolección de información y los debates.
	Capacidad de evaluación que evidentemente, teniendo en cuenta la necesidad de formar individuos capaces de tomar decisiones sustentables, es fundamental en cualquier programa que se emprenda. Puede lograrse mediante el análisis comparativo de distintas soluciones, la evaluación de acciones y sistemas, la simulación de situaciones, la organización de debates, etc.
	Participación, elemento vital y motivo primordial de la educación ambiental, alcanzada por medio de talleres de acción, actividades en la comunidad, simulación de situaciones complejas y juegos diversos.\\
	
	\section*{Importancia}
	Es necesario abordar la temática del cuidado del medio ambiente con la seriedad necesaria para poder revertir los hábitos que causaron daños, hasta la fecha, a nuestro planeta. Es necesario incorporar la idea que con el correr del tiempo y manteniendo comportamientos perjudiciales hacia el ambiente vamos perdiendo la oportunidad de tener una mejor calidad de vida, vamos deteriorando nuestro planeta y a los seres que habitan en él.
	
	Es evidente la necesidad de sensibilización desde cada uno de nosotros, para repensar en qué valores y actitudes, se acoda el cambio cultural que debemos asumir, con respecto a las problemáticas ambientales.
	
	Más allá de la educación tradicional, es decir, del simple hecho de impartir un conocimiento, la educación ambiental relaciona al hombre con su ambiente, con su entorno y busca un cambio de actitud, una toma de conciencia sobre la importancia de conservar para el futuro y para mejorar nuestra calidad de vida.
	
	La adopción de una actitud consciente ante el medio que nos rodea, y del cual formamos parte indisoluble, depende en gran medida de la enseñanza y la educación de la niñez y la juventud. Por esta razón, corresponde a la pedagogía y a la escuela desempeñar un papel fundamental en este proceso.
	
	La educación ambiental es un proceso educativo, integral e interdisciplinario que considera al ambiente como un todo y que busca involucrar a la población en general en la identificación y resolución de problemas a través de la adquisición de conocimientos, valores, actitudes y habilidades, la toma de decisiones y la participación activa y organizada.
	
	\section*{Conciencia ambiental}
	 Hacer conciencia ambiental, es conocer nuestro ambiente, nuestro entorno, cuidarlo, protegerlo y conservarlo para que nuestros hijos también disfruten de un ambiente sano.
	 
	 Conciencia Ambiental, es el entendimiento que se tiene del impacto de nosotros los seres humanos en el entorno, es decir; entender como influyen las acciones que cometemos cada día en el ambiente y como eso afecta el futuro de nuestro espacio y nuestros hijos.
	 
	 Conciencia Ambiental por ejemplo: es entender que si yo, ciudadano común, derrocho o malgasto algún recurso natural, como puede ser el agua, mañana cuando quiera volver a utilizarlo ya no voy a poder, por no conservar y hacer conciencia en el uso racional de este recurso tan importante para la vida humana. No olvidemos que todos los recursos que nos brinda el ambiente son muy importantes y debemos hacer un uso racional de ellos. Eso es hacer conciencia ambiental.
	 
	 La Conciencia Ambiental, se logra con educación,  en todos los niveles de la sociedad, en todo momento y en todo lugar hay que educar para poder concientizar.

	\section*{Conclusiones}
	El mundo que hasta este momento hemos creado como resultado de nuestra forma de pensar, tiene problemas que no pueden ser resueltos pensando del modo en cuando los creamos.
	
	Es necesario cambiar nuestra forma de pensar y de actuar para poder modificar el destino al que estamos llevando al planeta. Es neceario un cambio social, político, económico y cultural para evitar que la crisis ecológica destruya finalmente a la humanidad.
	Cuidemos nuestro ambiente, la naturaleza, los recursos naturales. Dale un respiro al ambiente, deja de contaminar.
	Hoy la naturaleza clama por ayuda, extiéndele la mano. Un mundo verde y limpio es mundo mejor.
	
	\section*{Referencias}
	$
	Subgerencia Cultural del Banco de la Republica. (2015). El medio ambiente. Recuperado de: http://www.banrepcultural.org/blaavirtual/ayudadetareas/ciencias/medio_ambiente\\
	\\
	http://sedema.df.gob.mx/educacionambiental/index.php/en/educacion-ambiental/que-es-educacion-ambiental\\
	\\
	http://www.ecopibes.com/educadores/objetivos.htm\\
	\\
	www.ecoportal.net\\
	\\
	http://todosobreelmedioambiente.jimdo.com/\\
	$
\end{document}\frac{núm}{den}